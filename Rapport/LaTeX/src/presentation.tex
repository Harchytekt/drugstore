\section{Présentation du projet}
\label{sec:presentation}


\subsection{Introduction}
\label{sec:intro}

Dans le cadre du cours de \textbf{Programmation web}, il nous a été demandé de réaliser un site dynamique en PHP.\\
Dans le cadre d'un site dynamique, les données disponibles sur le site proviennent d'une base de données, ici, composée de trois tables.


\subsection{Énoncé}
\label{sec:enonce}

Le site doit comprendre une base de données composée de trois tables.\\

L'une d'elles est prévue pour la gestion des utilisateurs, représentés par, au minimum, un nom, un prénom, une adresse mail et un mot de passe.\\
L'adresse mail doit servir pour la connexion au site, et le mot de passe a une longueur minimale de 4 caractères.\\

Du côté des fonctionnalités, les utilisateurs doivent pouvoir se (dé)connecter, modifier leurs données, supprimer leur compte et contacter l'administrateur.\\
L'\textbf{administrateur}, utilisateur créé par nos soins, a accès à toutes les données de la base de données, peut ajouter et modifier du contenu, réinitialiser le mot de passe des utilisateurs et peut aussi les bannir.


\subsection{Description}
\label{sec:description}

Ce site proposera aux visiteurs de s'inscrire afin de créer une liste privée de médicaments et ce, d’après une liste de médicaments \textit{non-exhaustive}.\\
Cette liste sera établie par l'administrateur selon son envie et les demandes envoyées par les utilisateurs.\\

Chaque médicament comprendra: sa posologie, les effets néfastes, les contre-\\indications ainsi qu'un lien vers sa notice.\\

Les personnes n’ayant pas de compte auront accès à la liste des médicaments ne contenant que leur nom et leur posologie.

\newpage


\subsection{Langages}
\label{sec:langages}

Les deux langages principaux sont donc, vu l'énoncé, le \textbf{PHP} et \textbf{MySQL}.\\
Ceux-ci sont, bien entendu, accompagnés du HTML, CSS ainsi que du JavaScript \textit{(et jQuery)} pour la création du site.

\vspace{1cm}

\begin{figure}[!h]
\centering
\begin{minipage}[c]{0.4\textwidth}
  \centering
  \includegraphics[scale=0.15]
  {textures/images/tools/php.pdf}
\caption{Logo de PHP}\label{php}
\end{minipage} \qquad
\begin{minipage}[c]{0.4\textwidth}
  \centering
  \includegraphics[scale=0.15]
  {textures/images/tools/mysql.pdf}
\caption{Logo de MySQL}\label{mysql}
\end{minipage}
\end{figure}


\subsection{Outils}
\label{sec:outils}

D'après le cours de bases de données du premier quadrimestre et le système d'exploitation utilisé, l'outil choisi pour le serveur web ainsi que la base de données est \textbf{MAMP}.\\
En ce qui concerne l'éditeur de texte, c'est \textbf{Atom} qui a été retenu pour sa simplicité ainsi que pour la familiarité que je ressent envers lui.

\vspace{1cm}

\begin{figure}[!h]
\centering
\begin{minipage}[c]{0.4\textwidth}
  \centering
  \includegraphics[scale=0.15]
  {textures/images/tools/mamp.pdf}
\caption{Logo de MAMP}\label{mamp}
\end{minipage} \qquad
\begin{minipage}[c]{0.4\textwidth}
  \centering
  \includegraphics[scale=0.15]
  {textures/images/tools/atom.pdf}
\caption{Logo de Atom}\label{atom}
\end{minipage}
\end{figure}


%%% Local Variables:
%%% mode: latex
%%% TeX-master: t
%%% End:
