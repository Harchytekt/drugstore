\section{Conclusion}
\label{sec:conclusion}


\subsection{Résultat}
\label{sec:resultat}

Le site propose donc à chacun de créer un compte afin de posséder sa liste de médicaments.\\

Un utilisateur, identifié par son adresse mail et son nom d'utilisateur, peut modifier ses données depuis la page \textit{Compte}.\\
Il peut ainsi modifier son nom de famille, son prénom, son adresse mail et son mot de passe. Il a aussi la possibilité de supprimer son compte.\\

L'administrateur peut être contacté par les utilisateurs du site que ce soit pour ajouter un médicament ou pour réinitialiser leur mot de passe.\\ Il peut aussi bannir des utilisateurs.\\
En plus d'ajouter des médicaments, il a bien sûr la possibilité de les modifier, mais pas de les supprimer.
En effet, un utilisateur possédant un médicament sur le point d'être supprimé serait mécontent de le perdre.\\
Comme les autres utilisateurs, il peut changer ses données, mais il ne peut pas supprimer son compte, étant donné qu'il est l'administrateur.\\

Les visiteurs du site ne possédant pas de compte ont la possibilité de parcourir la liste des différents médicaments disponibles.\\

Il est à remarquer que, d'après les bonnes pratiques des bases de données, il ne faut pas supprimer les données mais les rendre invisibles aux utilisateurs.\\
Cette pratique est utilisée lors de la suppression d'un compte qui n'est pas supprimé, mais simplement désactivé.


\subsection{Différences avec le cahier des charges}
\label{sec:diff-cahier}

Lors de la création de la maquette du site, l'utilisation d'avatars pour identifier les utilisateurs s'est avérée moins intéressante et utile qu'escompté.\\
En effet, les utilisateurs n'ayant pas d'interactions entre-eux et ne communiquant avec l'administrateur que par mails, les avatars sont devenus superflus.\\

Lors de la conception de la version préliminaire, il est apparu que les effets néfastes et les contre-indications étaient souvent similaires.\\
J'ai alors pris la décision de les regrouper afin d'éviter les doublons.

%%% Local Variables:
%%% mode: latex
%%% TeX-master: t
%%% End: