\section{Site}
\label{sec:site}


\subsection{Arborescence}
\label{sec:arborescence}

Le site a été divisé en neuf dossiers pour plus de clarté.

\begin{itemize}

    \item[$\bullet$] Trois d’entre-eux ne sont pas liés au PHP:
    \begin{itemize}
    
         \item \textbf{img :} contient toutes les images du site.
         
         \item \textbf{js :} contient les différents scripts en JavaScript ainsi que jQuery.
         
         \item \textbf{style :} contient les différents fichiers contenant le CSS du site.
         
    \end{itemize}
    
    \item[$\bullet$] Ensuite, viennent les six autres qui contiennent le PHP:
    \begin{itemize}
    
         \item \textbf{actions :} contient les scripts php liés aux actions telles que la\\ (dés)activation d’un compte, l’ajout d’un médicament ainsi que la ré-initialisation d’un mot de passe.
         
         \item \textbf{connexion :} contient les scripts php liés à la connexion à la base de données ainsi que la (dé)connexion au site.
         
         \item \textbf{DB :} contient les scripts MySQL qui créent et initialisent la base de données.
         
         \item \textbf{forms :} contient les différents formulaires (avec leur version de vérification) tels que celui de l’ajout/modification d’un médicament, de\\ connexion, etc.
         
         \item \textbf{include :} contient les scripts php formant une partie des pages comme le pied de page et les différentes tables de la base de données.
         
         \item \textbf{insidePages :} les pages internes du site comme la page principale, celle de changement de mot de passe, etc.
         
    \end{itemize}
    
    \item[$\bullet$] Pour finir, l’index et la page d’erreur de connexion à la base de données se trouvent à la racine du site.

\end{itemize}

\newpage


\subsection{Arborescence détaillée}
\label{sec:arb-details}

Voici maintenant chaque dossier détaillé.


\subsubsection{Dossiers non liés au PHP}
\label{sec:folder-no-php}

Le dossier \textit{js} contient trois fichiers.\\

Le premier, \textbf{\textit{jquery-3.1.1.min.js}}, contient toutes les fonctions de jQuery.\\

Le second, \textbf{\textit{jquery.cslide.js}}, est une librairie permettant d’insérer facilement un slider \footnote{\url{http://callmenick.com/post/responsive-content-slider}}.\\
Il est utilisé pour afficher le contenu des tables 5 par 5 grâce à l’utilisation de boutons \guillemotleft \ Suivant \guillemotright \ et \guillemotleft \ Précédent \guillemotright.\\

Le troisième et dernier, \textbf{\textit{script.js}},contient trois fonctions:
\begin{enumerate}

    \item hideUsers(): qui cache l'onglet \textit{Utilisateurs} lorsqu'un utilisateur autre que l'administrateur est connecté.
    
    \item setCurrentTab(): qui sélectionne l'onglet courant.
    
    \item updateFooterBg(): qui change la couleur de fond du footer sur les autres navigateurs que Safari, qui permet d'avoir un fond flouté.
    
\end{enumerate}


\subsubsection{Dossiers PHP}
\label{sec:folder-php}

Le dossier \textbf{actions} est constitué de quatre fichiers PHP.

\begin{itemize}

    \item \textbf{\textit{activateUser.php}}, qui permet de désactiver le compte d’un utilisateur ou de le réactiver.\\
    Tous les utilisateurs, à l’exception de l’administrateur, peuvent désactiver leur compte.\\
    L’administrateur, lui, a le droit d’activer ou de bannir \textit{(désactiver)} le compte de n’importe quel utilisateur sauf le sien.
    
    \item \textbf{\textit{addMedToList.php}}: ce fichier reçoit donc les données d’un nouveau médicament qu’il ajoute à la base de données.
    
    \item \textbf{\textit{changeDBData.php}}: il a un fonctionnement proche du précédent, sauf qu’il modifie les données de l’utilisateur (nom de famille, prénom et adresse mail).
    
    \item \textbf{\textit{reinitPWD.php}}: appelé par l’administrateur, ce fichier réinitialise le mot de passe de l’utilisateur sélectionné en lui donnant une valeur par défaut \textit{(EpcCM98)}.

\end{itemize}

\newpage

Le dossier \textbf{connexion} en possède douze.

\begin{itemize}

    \item \textbf{\textit{connexion.php}} contient le script de connexion à la base de données.
    
    \item \textbf{\textit{dbInfos.php}} contient les données de connexion à la base de données: nom d’hôte, nom de la base, nom d’utilisateur, mot de passe et numéro de port.
    
    \item \textbf{\textit{login.php}} est le script appelé pour vérifier le login lors d’une tentative de connexion au site.
    
    \item \textbf{\textit{loginPage.php}}: c’est la page de connexion au site. C’est elle qui appelle le script précédent.
    
    \item \textbf{\textit{logout.php}}: comme son nom l’indique, il est utilisé pour déconnecter l’utilisateur du site.
    
    \item \textbf{\textit{signUp.php}}: son fonctionnement proche de celui de login.php lui permet d’enregistrer un nouvel utilisateur.
    
    \item \textbf{\textit{signuPage.php}}: c’est la page d’inscription au site. C’est elle qui appelle le script précédent.
    
    \item Les cinq suivant servent à vérifier les données des formulaires à l’aide de regex\footnote{\href{https://fr.wikipedia.org/wiki/Expression\_rationnelle}{Expression régulière (en anglais: \textbf{Reg}ular \textbf{ex}pression)}}.\\
    Il est aussi parfois nécessaire de vérifier des données dans la base de données comme le nom d’un médicament lors de son ajout afin de ne pas avoir de doublons.\\

\end{itemize}

Le dossier \textbf{BD} est le plus simple, car il ne contient que deux fichiers SQL:

\begin{itemize}

    \item \textbf{\textit{CreateDB.sql}} crée la base de données ainsi que les trois tables la composant.
    
    \item \textbf{\textit{InitDB.sql}} initialise la base de données en ajoutant trois utilisateurs dont l’administrateur; ajoute sept médicaments différents; et remplit la réserve des deux utilisateurs créés.

\end{itemize}

\newpage

Le dossier \textbf{forms} contient tous les formulaires du site.\\

En résumé, chaque formulaire existe en deux formes:
\begin{itemize}
    
    \item[$\bullet$] Le formulaire vierge qui, une fois soumis, est vérifié par l’un des scripts de vérification du dossier connexion.
    
    \item[$\bullet$] Le formulaire de vérification qui, si les données sont acceptées, appelle le script adéquat.\\
    Dans le cas contraire, le formulaire est affiché avec les bonnes données et/ou les messages d’erreurs pour chaque champ.\\
    
\end{itemize}

Le dossier \textbf{include} contient 8 fichiers.

\begin{itemize}
    
    \item \textbf{\textit{footer.php}} contient le footer avec ou sans le moyen de contacter l’administrateur selon que l’on soit connecté ou non et que l’on soit l’administrateur ou non.
    
    \item Les deux fichiers de barre de navigation:
    \begin{itemize}
        
        \item \textbf{\textit{navBarConnect.php}} permet d’afficher la liste des médicaments et les liens de connexion et d’inscription au site.
        
        \item \textbf{\textit{navBarReserve.php}}: affichée lorsque l’utilisateur est connecté, elle permet de naviguer entre les différentes tables de la base.
        
    \end{itemize}
    
    \item \textbf{\textit{medsPopUp.php}} permet d’afficher les données des médicaments dans un popup.
    
    \item \textbf{\textit{account.php}} affiche les données de l’utilisateur et lui permet de les changer ou de supprimer son compte.
    
    \item Les trois derniers permettent d’afficher les données contenues dans les différentes tables d’après l’utilisateur connecté.
    
\end{itemize}

\newpage

Le dernier dossier est intitulé \textbf{insidePages}, car il est constitué des différentes pages disponibles lorsque l'utilisateur est connecté.

\begin{itemize}
    
    \item \textbf{\textit{chgMed.php}} est la page qui permet d’ajouter ou de modifier un médicament à la base.
    
    \item \textbf{\textit{chgPwd.php}} permet de changer de mot de passe.
    
    \item \textbf{\textit{chooseImage.php}} propose de choisir une image \textit{(png)} pour son médicament.
    
    \item \textbf{\textit{page.php}} est la page la plus utilisée du site: c’est elle qui permet de consulter les données de la base, de changer ses données personnelles et de supprimer son compte.
    
    \item \textbf{\textit{productList.php}} est la page qui permet de consulter la liste des médicaments sans être connecté.
    
\end{itemize}


\subsubsection{Dossier racine}
\label{sec:folder-root}

Le fichier \textbf{\textit{index.php}} affiche, comme on peut s’en douter, la page d’accueil du site qui permet de se connecter et de s’inscrire au site, et d’afficher la liste des médicaments.\\

La page d’erreur de connexion à la base de données, \textbf{\textit{DBerror.php}}, affiche un message d’erreur \textit{(erreur 503)} lorsqu’il y a une erreur de connexion à la base de données.

\newpage


\subsection{Connexion à la base de données}
\label{sec:conn-bd}

En ce qui concerne la connexion à la base de données, aucun \textit{design pattern} n’a été utilisé, car, lorsque cette matière a été abordée au cours, la majeure partie du projet était déjà achevée.\\

Le type de requêtes utilisé pour l'accès à la base de données est le type PDO.\\
De manière plus précise, ce sont des requêtes préparées qui ont été utilisées pour plus de sécurité \textit{(voir la section~\ref{sec:securite}\textbf{~\nameref{sec:securite}})}.\\
De plus, PDO permet plus de flexibilité, est recommandée, très utilisée \textit{(et possède donc une importante communauté)} et est enseignée à l'école.\\

On pourrait alors ajouter, dans les évolutions possibles, le passage à la \textit{programmation orientée objet}, couplée à l’utilisation du \textit{singleton} et, comme structure de site, l’utilisation du \textit{design pattern \textbf{MVC}}.

\newpage


\subsection{Sécurité}
\label{sec:securite}

Pour la sécurité du site tout accès à la base de données passe par des requêtes préparées \textit{(PDO)}, car celles-ci protègent le site de toute injection SQL.\\

La connexion au site est surveillée grâce à la mise en place d’une session. Dès qu’un accès à une page se fait, la connexion est vérifiée. Si l’utilisateur n’est pas identifié, il est redirigé vers la page d’accueil.\\

En plus de cela, chaque champ de formulaire est protégé par des expressions régulières \textit{(regex)}.\\
Celles-ci vérifient que les données entrées sont de la forme voulue: un nom ne doit contenir que des lettres et, selon une certaine forme, peut être divisé à l’aide d’un trait d’union; une adresse mail ne contient que des lettres minuscules non-accentuées, des points et un seul arobase; etc.\\

De plus, les mots de passe ont une longueur minimale de quatre caractères et doivent contenir, au moins, une majuscule, un minuscule et un chiffre.\\
Ils sont aussi hachés à l’aide de \textit{SHA512} et d’un \textit{salage dynamique} pour plus de sécurité.\\
En effet, le mot de passe est de la forme \textbf{+\%\# longueur mot\_de\_passe  ¨*§}.\\ Par exemple, le mot de passe \textbf{Test1} devient \textbf{+\%\#5Test1¨*§} avant d’être haché.\\
Il est ainsi indéchiffrable, mais cela a aussi pour effet de sécuriser la base de données puisque le résultat ne contiendra que des caractères sûrs.

%%% Local Variables:
%%% mode: latex
%%% TeX-master: t
%%% End:
